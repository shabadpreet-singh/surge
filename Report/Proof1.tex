
% \documentclass{article}
% \usepackage[english]{babel}
% \usepackage[a4paper,margin=1in,footskip=0.25in]{geometry}
% \usepackage{amssymb,amsmath,latexsym,color,amsthm,authblk,mathpazo,palatino,bbding,setspace,datetime,breakcites,hyphenat,microtype,diagbox,tikz,xcolor,tkz-graph,subcaption,graphicx,titlesec,amsbsy, listings}
% \usepackage{titlesec}
% \usepackage{mathptmx}
% %\usepackage{times}
% %\usepackage{cite}
% \usepackage[round,authoryear]{natbib}
% \citestyle{authordate}
% \usepackage{hyperref,theoremref}
% \usepackage[titletoc,title]{appendix}
% %\usepackage[author={Souvik Roy}]{pdfcomment}
% %\usepackage{parskip}
% \usepackage[shortlabels]{enumitem}
% \usepackage{float}
% \usepackage{amssymb}
% \restylefloat{table}
% \everymath{\displaystyle}
% \usepackage{pstricks}

% \usepackage{tikz}
% \usetikzlibrary{arrows}


% %\usepackage{accents}
% %\setlength{\parindent}{15pt}
% \usepackage{accents}

% \newcommand{\ut}[1]{\underaccent{\tilde}{#1}}
% \renewcommand{\vec}[1]{\ut{#1}}

% \definecolor{armygreen}{rgb}{0.29, 0.8, 0.13}
% \definecolor{auburn}{rgb}{0.43, 0.21, 0.1}
% \definecolor{burgundy}{rgb}{0.5, 0.0, 0.13}
% \definecolor{medium red}{rgb}{.490,.298,.337}
% \definecolor{dark red}{rgb}{.235,.141,.161}

% \hypersetup{
% 	colorlinks = true,
% 	linkcolor = {burgundy},
% 	citecolor = {burgundy}, 		
% 	linkbordercolor = {white},
% }

% \captionsetup[sub]{font=scriptsize}

% % Page length commands go here in the preamble
% %\setlength{\oddsidemargin}{-0.25in} % Left margin of 1 in + 0 in = 1 in
% %\setlength{\textwidth}{7in}   % Right margin of 8.5 in - 1 in - 6.5 in = 1 in
% %\setlength{\topmargin}{-.75in}  % Top margin of 2 in -0.75 in = 1 in
% %\setlength{\textheight}{9.5in}  % Lower margin of 11 in - 9 in - 1 in = 1 in

% \interfootnotelinepenalty=10000
% \raggedbottom

% \let\OLDthebibliography\thebibliography
% \renewcommand\thebibliography[1]{
% 	\OLDthebibliography{#1}
% 	\setlength{\parskip}{0pt}
% 	\setlength{\itemsep}{0pt plus 0.1ex}
% }

% \newcommand{\va}{\vartriangleleft}
% \newcommand{\vaq}{\trianglelefteq}
% \newcommand{\rva}{\vartriangleright}
% \newcommand{\rvaq}{\trianglerighteq}


% \DeclareFontFamily{U}{mathx}{\hyphenchar\font45}
% \DeclareFontShape{U}{mathx}{m}{n}{<-> mathx10}{}
% \DeclareSymbolFont{mathx}{U}{mathx}{m}{n}
% \DeclareMathAccent{\widebar}{0}{mathx}{"73}

% \titleformat{\section}[block]{\normalfont\scshape\large\filcenter}{\thesection .}{1em}{}
% \titleformat{\subsection}{\normalfont\scshape\large}{\thesubsection}{1em}{}
% \titleformat{\subsubsection}{\normalfont\scshape\large}{\thesubsubsection}{1em}{}
% \titleformat{\paragraph}
% {\normalfont\scshape\large}{\theparagraph}{1em}{}
% \titlespacing*{\paragraph}
% {0pt}{3.25ex plus 1ex minus .2ex}{1.5ex plus .2ex}

% % Theorem Styles
% \newtheorem{theorem}{Theorem}
% \newtheorem{proposition}{Proposition}
% \newtheorem{claim}{Claim}
% \newtheorem{lemma}{Lemma}
% \newtheorem{corollary}{Corollary}
% \newtheorem{observation}{Observation}[section]
% % Definition Styles
% \theoremstyle{definition}
% \newtheorem{definition}{Definition}[section]
% \newtheorem{example}{Example}[section]
% \theoremstyle{remark}
% \newtheorem{remark}{\textsc{Remark}}[section]

% \renewenvironment{proof}[1][\proofname]{{\bfseries #1: }}{\qed}
% \newcommand{\tbf}{\textbf}
% \newcommand{\sr}{\textcolor{red}}
% \newcommand{\srs}{\textcolor{blue}}
% \newcommand{\srsk}{\textcolor{armygreen}}
% \newdateformat{monthyeardate}{\monthname[\THEMONTH], \THEYEAR}
%\begin{document}

\begin{proof}
Let $(\succ,\rva)$ be a preference profile where for all $i\in \{1,2,3\}$, $M_i$ has preference $\succ_i$ and $W_i$ has preference $\rva_i$, and $M_1$ is matched to her third preferred women in DA. Without loss of generality, let's assume $W_1 \succ_1 W_2 \succ_1 W_3$, and $M_2$ and $M_3$ are matched with $W_1$ and $W_2$, respectively in DA. Since the outcome of DA is stable, $W_1$ must prefer $M_2$ over $M_1$ and $W_2$ must prefer $M_3$ over $M_1$. Therefore, we have \[  \begin{array}{c}
	   \rva_1 \\
	\hline
	\vdots \\
	M_2\\
        \vdots\\
        M_1\\
        \vdots
\end{array} 
\hspace{10mm}
%
\begin{array}{c}
	\rva_2 \\
	\hline
	\vdots \\
	M_3\\
        \vdots\\
        M_1\\
        \vdots
\end{array} 
\]
Further, $M_2$ will prefer $W_1$ over $W_3$ and $M_3$ will prefer $W_2$ over $W_3$. To see this, assume for contradiction  $M_2$ prefers $W_3$ over $W_1$. Since finally $M_2$ is matched with $W_1$, it means at some stage of the algorithm (say $t^{th}$ stage), $M_2$'s proposal to $W_3$ got rejected. Suppose $M_1$'s proposal to $W_3$ caused this rejection. Since $W_3$ is $M_1$'s least preferred woman, it must be that he had already proposed to $W_1$ and $W_2$. Therefore, by DA algorithm, $W_1$ and $W_2$ had at least one proposal at $t^{th}$ stage. But this is a contradiction as $W_3$ had two proposals at $t^{th}$ stage. Now assume $M_3$'s proposal to $W_3$ caused the rejection of $M_2$'s proposal to $W_3$ at the $t^{th}$ stage. Since finally $M_3$ is matched with $W_2$, it means, at some later stage, his proposal to $W_3$ also got rejected. But this rejection can only cause due to $M_1$'s proposal to $W_3$, which again means, at that stage, $W_3$ had two proposals, a contradiction. Thus, we have

% Consider the following matching, $M_1$ is matched with $W_1$, $M_2$ is matched with $W_3$, and $M_3$ is matched with $W_2$. We claim that this matching is stable. As $M_1$ is getting his top-women, he will not be in a blocking pair. Suppose $M_2$ blocks with $W_2$. This means $M_2$ prefers $W_2$ over $W_3$, hence the most, and $W_2$ prefers $M_2$ over $M_3$, hence, the most. But this contradicts as they are not matched in DA. Finally, suppose $M_3$ blocks with either $W_1$ or $W_3$. 


\[  \begin{array}{c}
	\succ_2 \\
	\hline
	\vdots \\
	W_1\\
        \vdots\\
        W_3\\
        \vdots
\end{array} 
\hspace{10mm}
%
\begin{array}{c}
	\succ_3 \\
	\hline
	\vdots \\
	W_2\\
        \vdots\\
        W_3\\
        \vdots
\end{array} 
\]
We claim that, in TTC, $M_1$ will be matched with $W_3$. Note that $M_2$ and $M_3$ have either $W_1$ or $W_2$ as their top preferred women. Moreover, $W_1$ and $W_2$ have either $M_2$ or $M_3$ as their top-preferred men. Therefore, in the first step of TTC, either they will point towards each other, or they will point towards themselves, or they both will point to either $M_2$ or $M_3$. 
If they both point to each other or themselves, they will get matched to $W_1$ and $W_2$, and as a result, $M_1$ will go with $W_3$. So, assume they both point to one of them. Since, in DA, $M_2$ got matched with $W_1$ and $M_3$ is matched with $W_2$, this means either both point to $M_2$ and $W_1$ is with $M_2$ initially, or both point to $M_3$ and $W_2$ is with $M_3$ initially. Suppose both point to $M_2$ and $W_1$ is with $M_2$ initially. Hence, $M_2$ will be matched with $W_1$. Further, as $M_3$ prefers $W_2$ over $W_3$ and $W_2$ prefers $M_3$ over $M_1$, $M_3$ will be matched to $W_2$ in the second round implying $M_1$ will be matched with $W_3$. The argument is similar when both point to $M_3$ and $W_2$ is with $M_3$ initially. This completes the proof of the lemma.
\end{proof}

% Suppose not, and first assume that $M_1$ is matched with $W_1$. Since $W_1$ prefers $M_2$ over $M_1$, by the TTC algorithm,  it necessarily means $M_2$ is matched with someone whom he prefers over $W_1$. Thus, $M_2$ is matched with $W_2$ in TTC, and hence, $M_3$ is matched with $W_3$. Also, as $M_3$ prefers $W_2$ over $W_3$, his match in the algorithm, it must be that $W_2$ prefers her match over $M_3$. This means $M_2\rva_2 M_3$. Combining all these observations, we have

% \[  \begin{array}{c}
% 	   \rva_2 \\
% 	\hline
% 	M_2\\
%         M_3\\
%         M_1\\
% \end{array} 
% \hspace{10mm}
% %
% \begin{array}{c}
% 	\succ_2 \\
% 	\hline
% 	W_2 \\
% 	W_1\\
%         W_3\\
% \end{array} 
% \]

% % and \[  \begin{array}{c}
% % 	\succ_2 \\
% % 	\hline
% % 	\vdots \\
% % 	W_1\\
% %         \vdots\\
% %         W_3\\
% %         \vdots
% % \end{array} 
% % \hspace{10mm}
% % %
% % \begin{array}{c}
% % 	\succ_3 \\
% % 	\hline
% % 	\vdots \\
% % 	W_2\\
% %         \vdots\\
% %         W_3\\
% %         \vdots
% % \end{array} 
% % \]


% % $M_1$, $M_2$, and $M_3$ have preferences $\succ_1$, $\succ_2$, and $\succ_3$, respectively, and $W_1$, $W_2$, and $W_3$ have preferences $\rva_1$, $\rva_2$, and $\rva_3$, respectively. Without loss of generality $W_1 \succ_1 W_2 \succ_1 W_3$.


% Without loss of generality assume that $M_1$ has the preference $\succ_1$ where  $W_1 \succ_1 W_2 \succ_1 W_3$. If $M_1$ is getting his third preferred woman ($W_3$) in DA, it means he got rejected by the top two women in his preference in the algorithm. Therefore, three cases are possible. We consider them separately below.

% \noindent\textbf{Case I :-} All men have  $W_1$ as their first preferred women.

% Now applying DA;
% $$\textbf{Step 1}: M_1, M_2, M_3 \text{ proposes } W_1$$
% For $W_1$ to reject $M_1$, one of ${M_2, M_3}$ must be above $M_1$ in her preference profile.  Without loss of generality, assume $M_2$ to be  the first preference of  $W_1$.  Therefore preference profile of 
% \begin{equation}
% W_1 = (M_2 \succ ... ) \label{eq1}
% \end{equation}

% \begin{center}$\therefore$ $W_1$ reject both $M_1$ and $M_3$.\end{center}

% Now, for $M_1$ to get rejected by his $2^{nd}$ preferred women $M_3$ must also propose $W_2$ and get accepted by her. Therefore, preference profile of  \begin{equation}W_2 = (... M_3 \succ ... M_1 \succ...)\label{eq2}\end{equation}

% \begin{center}\textbf{Step 2 :}$W_2$ accepts $M_3$ and rejects $M_1$\end{center}
% $$\textbf{Step 3 :} M_1 \text{ proposes } W_3$$

% Not getting any other proposals, she accepts $M_1$, and $M_1$ gets paired with his third preferred woman.

% By using \eqref{eq1} and \eqref{eq2}, we can make a preference profile table of all men and women; 
% \begin{center}
% \begin{tabular}{ c|c|c } 
 
%  $M_1$ & $M_2$ & $M_3$  \\ 
% \hline
%  $W_1$ & $W_1$ & $W_1$ \\ 
%  $W_2$ & : & $W_2$ \\ 
%  $W_3$ & : & :\\

% \end{tabular}
% \begin{tabular}{ c|c|c } 
 
%  $W_1$ & $W_2$ & $W_3$  \\ 
% \hline
%  $M_2$ & : & :\\ 
%  : & $M_3$ & :\\ 
%  : & : & :\\
%   & $M_1$ & \\
%   & : & \\

% \end{tabular}
% \end{center}
% Now, we apply TTC by constructing a directed graph with 3 vertices, one for each man and putting a directed edge from vertex of men i to vertex of men j if the top-ranked object of men i is endowed with
% men j.\\
% In each graph, men who are part of the \textit{cycle with blue edges} can trade their endowments.
% \\
% \begin{center}
% \begin{tikzpicture}[->,>=stealth',auto,node distance=3cm,
%   thick,main node/.style={circle,draw,font=\sffamily\Large\bfseries}]

%   \node[main node] (1) {$M_1$};
%   \node[main node] (2) [below right of=1] {$M_2$ $[W_1]$};
%   \node[main node] (3) [above right of=2] {$M_3$};
  
%  %\node[main node] (4) [right of=3] {d};

% % [->,>=stealth',auto,node distance=1mm,
% %   thick,main node/.style={[],draw,font=\sffamily\small}]

% %   \node[main node] (4) [below of=2] {$W_2$};

%   \path[every node/.style={font=\sffamily\small}]
%     (1) edge node [right] {} (2)
%     (2) edge[blue] [loop right] node [right] {} (2)
%     (3) edge node [left] {} (2);
% %    (3) edge node [right] {} (4)
%  %   (4) edge[bend right] node [left] {} (1);
% % \node [below=1.5cm] at (2)
%  %       {\textbf{Step 1} };
% \end{tikzpicture}
% \end{center}
% Here, we see that there is only one cycle involving
% $M_2$. Therefore, $M_2$ gets paired with $W_1$.
% \\

% \begin{center}
% \begin{tikzpicture}[->,>=stealth',auto,node distance=3cm,
%   thick,main node/.style={circle,draw,font=\sffamily\Large\bfseries}]

%   \node[main node] (1) {$M_1$};
%   \node[main node] (3) [right of=1] {$M_3[W_2]$};
  
%  %\node[main node] (4) [right of=3] {d};

% % [->,>=stealth',auto,node distance=1mm,
% %   thick,main node/.style={[],draw,font=\sffamily\small}]

% %   \node[main node] (4) [below of=2] {$W_2$};

%   \path[every node/.style={font=\sffamily\small}]
%     (1) edge node [right] {} (3)
%     (3) edge[blue, loop right] node [right ] {} (3);
% %    (3) edge node [right] {} (4)
%  %   (4) edge[bend right] node [left] {} (1);
% % \node [below=1.5cm] at (2)
%  %       {\textbf{Step 1} };
% \end{tikzpicture}
% \end{center}
% Here also, we see that there is only one cycle involving
% $M_3$. Therefore, $M_3$ gets paired with $W_2$.
% \\
% $\therefore$ With no other choice left \textbf{$M_1$ gets paired to $W_3$} (\textit{his third preference}).
% \\ \\


% \textbf{Case II :-} $M_2$ and $M_3$ have $M_1$'s second preffered woman( i.e. $W_2$) as their first preference.

% Now, applying DA;
% $$
% \textbf{Step 1}
% \begin{cases}
% {M_1} \text{ proposes} W_1 \\
% {M_2, M_3} \text{ proposes} W_2
% \end{cases}
% $$
% Without loss of generality, let us assume that $M_3$ gets rejected by $W_2$, which means $M_3$ is somewhere below $M_2$ in $W_2$ preference. i.e.  \begin{equation}W_2 = (M_2 \succ ...)\label{eq3}\end{equation}

% Now if $M_3$ were to propose $W_2$, $M_1$ will be paired with $W_1$, which is not something we want, so in order for $M_1$ to form pair with $W_3$, $M_3$ must propose $W_1$ ans she must accept his proposal. 

% Therefore, preference profile of \begin{equation}W_1 = (... M_3 \succ ... M_1 \succ...)\label{eq4}\end{equation}

% $$\textbf{Step 2 :} M_3 \text{ proposes }W_1$$
% $\therefore$$W_1$ accepts $M_3$ and rejects $M_1$

% $$\textbf{Step 3 :}M_1 \text{ proposes }W_2$$
% As $W_2$ already got a proposal from her top pref $M_2$, therefore she'll reject $M_1$.
% $$\textbf{Step 4 :} M_1 \text{ proposes } W_3$$
% With no other proposals, she'll accept $M_1$. With this, $M_1$ is finally paired with his third preference, $W_3$.

% By using \eqref{eq3} and \eqref{eq4}, we can make a preference profile table of all men and women; 
% \begin{center}
% \begin{tabular}{ c|c|c } 
 
%  $M_1$ & $M_2$ & $M_3$  \\ 
% \hline
%  $W_1$ & $W_2$ & $W_2$ \\ 
%  $W_2$ & : & $W_1$ \\ 
%  $W_3$ & : & :\\

% \end{tabular}
% \begin{tabular}{ c|c|c } 
 
%  $W_1$ & $W_2$ & $W_3$  \\ 
% \hline
%  : &  & \\ 
%  $M_3$ & $M_2$ & :\\ 
%  : & : & :\\
%   $M_1$& : & :\\
%  : &  & \\

% \end{tabular}
% \end{center}
% \\
% Now, we apply TTC by constructing a directed graph with 3 vertices, one for each man and putting a directed edge from vertex of men i to vertex of men j if the top-ranked object of men $i$ is endowed with
% men $j$.

% In each graph, men who are part of the \textit{cycle with blue edges} can trade their endowments.
% \\


% \begin{center}
% \begin{tikzpicture}[->,>=stealth',auto,node distance=3cm,
%   thick,main node/.style={circle,draw,font=\sffamily\Large\bfseries}]

%   \node[main node] (1) {$M_1$};
%   \node[main node] (2) [below right of=1] {$M_2$ $[W_2]$};
%   \node[main node] (3) [above right of=2] {$M_3[W_1]$};
  
%  %\node[main node] (4) [right of=3] {d};

% % [->,>=stealth',auto,node distance=1mm,
% %   thick,main node/.style={[],draw,font=\sffamily\small}]

% %   \node[main node] (4) [below of=2] {$W_2$};

%   \path[every node/.style={font=\sffamily\small}]
%     (1) edge[bend left] node [ right] {} (3)
%     (2) edge[blue] [loop left] node [left] {} (2)
%     (3) edge node [left] {} (2);
% %    (3) edge node [right] {} (4)
%  %   (4) edge[bend right] node [left] {} (1);
% % \node [below=1.5cm] at (2)
%  %       {\textbf{Step 1} };
% \end{tikzpicture}
% \end{center}
% It doesn't matter among $M_3, M_2$ who is at top preference of $W_1$ as we see that there is only one cycle involving $M_2$. Therefore, $M_2$ gets paired with $W_1$.

% \begin{center}
% \begin{tikzpicture}[->,>=stealth',auto,node distance=3cm,
%   thick,main node/.style={circle,draw,font=\sffamily\Large\bfseries}]

%   \node[main node] (1) {$M_1$};
%   \node[main node] (3) [right of=1] {$M_3[W_1]$};
  
%  %\node[main node] (4) [right of=3] {d};

% % [->,>=stealth',auto,node distance=1mm,
% %   thick,main node/.style={[],draw,font=\sffamily\small}]

% %   \node[main node] (4) [below of=2] {$W_2$};

%   \path[every node/.style={font=\sffamily\small}]
%     (1) edge node [right] {} (3)
%     (3) edge[blue, loop right] node [right] {} (3);
% %    (3) edge node [right] {} (4)
%  %   (4) edge[bend right] node [left] {} (1);
% % \node [below=1.5cm] at (2)
%  %       {\textbf{Step 1} };
% \end{tikzpicture}
% \end{center}
% Here also, we see that there is only one cycle involving $M_3$. Therefore, $M_3$ gets paired with $W_1$.\\
% $\therefore$ With no other choice left \textbf{$M_1$ gets paired to $W_3$}(\textit{his third preference}).
% \\
% \\
% \textbf{Case III :- } One of the men has $W_1$ as his first preference and other has $W_2$ as his first preference.

% Without loss of generality, assuming the preferences of all men be:

% \begin{center}
%     \begin{tabular}{c|c|c}
%         $M_1$ & $M_2$ & $M_3$ \\
%         \hline
%         $W_1$ & $W_1$ & $W_2$ \\
%         $W_2$ & :   &   : \\
%         $W_3$ & :   &   :
%     \end{tabular}
% \end{center}

% A few things required to have $M_1$ get paired with his third preference are:
% \begin{itemize}
%     \item $W_2$ must prefer $M_3$ over $M_1$ as to reject $M_1$ later.
%     \item $W_1$ must prefer $M_2$ over $M_1$ as to reject $M_1$ at forst step of DA.
% \end{itemize}
% Keeping above things in mind preferences of women will look like:

% \begin{center}
%     \begin{tabular}{c|c|c}
%         $W_1$ & $W_2$ & $W_3$ \\
%         \hline
%           :   &   :   &   \\
%         $M_2$ & $M_3$ & : \\
%           :   &   :   & : \\
%         $M_1$ & $M_1$ & : \\
%           :   &   :   &   
%     \end{tabular}
% \end{center}

% On the basis of preferences of $W_1$ and $W_2$, assignment through TTC will progress differently, so we will make cases and see the result of TTC over different preferences of $W_1$ and $W_2$  :

% \textbf{a)}
% \begin{center}
%     \begin{tabular}{c|c|c}
%         $W_1$ & $W_2$ & $W_3$ \\
%         \hline
%           $M_3$   &  $M_2$   & :  \\
%         $M_2$ & $M_3$ & : \\
%         $M_1$ & $M_1$ & : \\
%     \end{tabular}
% \end{center}

% Applying TTC;

% \begin{center}
% \begin{tikzpicture}[->,>=stealth',auto,node distance=3cm,
%   thick,main node/.style={circle,draw,font=\sffamily\Large\bfseries}]

%   \node[main node] (1) {$M_1$};
%   \node[main node] (2) [below right of=1] {$M_2[W_2]$};
%   \node[main node] (3) [above right of=2] {$M_3[W_1]$};
  
%  %\node[main node] (4) [right of=3] {d};

% % [->,>=stealth',auto,node distance=1mm,
% %   thick,main node/.style={[],draw,font=\sffamily\small}]

% %   \node[main node] (4) [below of=2] {$W_2$};

%   \path[every node/.style={font=\sffamily\small}]
%     (1) edge[bend left] node [ right] {} (3)
%     (2) edge[blue] [bend left] node [right] {} (3)
%     (3) edge[blue, bend left] node [right] {} (2);
% %    (3) edge node [right] {} (4)
%  %   (4) edge[bend right] node [left] {} (1);
% % \node [below=1.5cm] at (2)
%  %       {\textbf{Step 1} };
% \end{tikzpicture}
% \end{center}
% A cycle exists between $M_2 \& M_3$. Therefore, $M_2$ gets paired to $W_1$, while  $M_3$ gets paired to $W_2$.
% \\ \\
% The only women left unmatched is $W_3$, so \textbf{$M_1$ gets paired with $W_3$}.\\

% \textbf{b)}
% \begin{center}
%     \begin{tabular}{c|c|c}
%         $W_1$ & $W_2$ & $W_3$ \\
%         \hline
%         $M_3$ & $M_3$   & :  \\
%         $M_2$ & : & : \\
%         $M_1$ & : & : \\
%     \end{tabular}
% \end{center}

% Applying TTC;
% \begin{center}
% \begin{tikzpicture}[->,>=stealth',auto,node distance=3cm,
%   thick,main node/.style={circle,draw,font=\sffamily\Large\bfseries}]

%   \node[main node] (1) {$M_1$};
%   \node[main node] (2) [below right of=1] {$M_2$};
%   \node[main node] (3) [above right of=2] {$M_3[W_1, W_2]$};
  
%  %\node[main node] (4) [right of=3] {d};

% % [->,>=stealth',auto,node distance=1mm,
% %   thick,main node/.style={[],draw,font=\sffamily\small}]

% %   \node[main node] (4) [below of=2] {$W_2$};

%   \path[every node/.style={font=\sffamily\small}]
%     (1) edge node [ right] {} (3)
%     (2) edge node [right] {} (3)
%     (3) edge[blue, loop right] node [right] {} (2);
% %    (3) edge node [right] {} (4)
%  %   (4) edge[bend right] node [left] {} (1);
% % \node [below=1.5cm] at (2)
%  %       {\textbf{Step 1} };
% \end{tikzpicture}
% \end{center}
% Here, we see there is only one cycle involving $M_3$. Therefore, $M_3$ gets paired with $W_2$.

% \begin{center}
% \begin{tikzpicture}[->,>=stealth',auto,node distance=3cm,
%   thick,main node/.style={circle,draw,font=\sffamily\Large\bfseries}]

%   \node[main node] (1) {$M_1$};
%   \node[main node] (2) [right of=1] {$M_2[W_1]$};
  
%  %\node[main node] (4) [right of=3] {d};

% % [->,>=stealth',auto,node distance=1mm,
% %   thick,main node/.style={[],draw,font=\sffamily\small}]

% %   \node[main node] (4) [below of=2] {$W_2$};

%   \path[every node/.style={font=\sffamily\small}]
%     (1) edge node [ right] {} (2)
%     (2) edge[blue, loop right] node [right] {} (2);
% %    (3) edge node [right] {} (4)
%  %   (4) edge[bend right] node [left] {} (1);
% % \node [below=1.5cm] at (2)
%  %       {\textbf{Step 1} };
% \end{tikzpicture}
% \end{center}
% Here, a cycle involving $M_2$ exists. Therefore, $M_2$ gets paired with $W_1$.
% \\ \\
% The only women left unmatched is $W_3$, so \textbf{$M_1$ gets paired with $W_3$}.\\

% \textbf{c)}
% \begin{center}
%     \begin{tabular}{c|c|c}
%         $W_1$ & $W_2$ & $W_3$ \\
%         \hline
%           $M_2$   & $M_2$   & :  \\
%                 : & $M_3$ & : \\
%                 : & $M_1$ & : \\
%     \end{tabular}
% \end{center}


% Applying TTC;
% \begin{center}
% \begin{tikzpicture}[->,>=stealth',auto,node distance=3cm,
%   thick,main node/.style={circle,draw,font=\sffamily\Large\bfseries}]

%   \node[main node] (1) {$M_1$};
%   \node[main node] (2) [below right of=1] {$M_2[W_1, W_2]$};
%   \node[main node] (3) [above right of=2] {$M_3$};
  
%  %\node[main node] (4) [right of=3] {d};

% % [->,>=stealth',auto,node distance=1mm,
% %   thick,main node/.style={[],draw,font=\sffamily\small}]

% %   \node[main node] (4) [below of=2] {$W_2$};

%   \path[every node/.style={font=\sffamily\small}]
%     (1) edge node [ right] {} (2)
%     (2) edge[blue, loop right] node [right] {} (2)
%     (3) edge node [right] {} (2);
% %    (3) edge node [right] {} (4)
%  %   (4) edge[bend right] node [left] {} (1);
% % \node [below=1.5cm] at (2)
%  %       {\textbf{Step 1} };
% \end{tikzpicture}
% \end{center}
% Here, we see there is only one cycle involving $M_2$. Therefore, $M_2$ gets paired with $W_1$.

% \begin{center}
% \begin{tikzpicture}[->,>=stealth',auto,node distance=3cm,
%   thick,main node/.style={circle,draw,font=\sffamily\Large\bfseries}]

%   \node[main node] (1) {$M_1$};
%   \node[main node] (3) [right of=1] {$M_3[W_2]$};
  
%  %\node[main node] (4) [right of=3] {d};

% % [->,>=stealth',auto,node distance=1mm,
% %   thick,main node/.style={[],draw,font=\sffamily\small}]

% %   \node[main node] (4) [below of=2] {$W_2$};

%   \path[every node/.style={font=\sffamily\small}]
%     (1) edge node [ right] {} (3)
%     (3) edge[blue, loop right] node [right] {} (3);
% %    (3) edge node [right] {} (4)
%  %   (4) edge[bend right] node [left] {} (1);
% % \node [below=1.5cm] at (2)
%  %       {\textbf{Step 1} };
% \end{tikzpicture}
% \end{center}
% Here, a cycle involving $M_3$ exists. Therefore, $M_3$ gets paired with $W_2$.
% \\ \\
% The only women left unmatched is $W_3$, so \textbf{$M_1$ gets paired with $W_3$}.\\

% \textbf{d)}
% \begin{center}
%     \begin{tabular}{c|c|c}
%         $W_1$ & $W_2$ & $W_3$ \\
%         \hline
%           $M_2$   & $M_3$   & :  \\
%                 : & : & : \\
%                 : & : & : \\
%     \end{tabular}
% \end{center}

% Applying TTC;
% \begin{center}
% \begin{tikzpicture}[->,>=stealth',auto,node distance=3cm,
%   thick,main node/.style={circle,draw,font=\sffamily\Large\bfseries}]

%   \node[main node] (1) {$M_1$};
%   \node[main node] (2) [below right of=1] {$M_2[W_1]$};
%   \node[main node] (3) [above right of=2] {$M_3[W_2]$};
  
%  %\node[main node] (4) [right of=3] {d};

% % [->,>=stealth',auto,node distance=1mm,
% %   thick,main node/.style={[],draw,font=\sffamily\small}]

% %   \node[main node] (4) [below of=2] {$W_2$};

%   \path[every node/.style={font=\sffamily\small}]
%     (1) edge node [ right] {} (2)
%     (2) edge[blue, loop right] node [right] {} (2)
%     (3) edge[blue, loop right] node [right] {} (3);
% %    (3) edge node [right] {} (4)
%  %   (4) edge[bend right] node [left] {} (1);
% % \node [below=1.5cm] at (2)
%  %       {\textbf{Step 1} };
% \end{tikzpicture}
% \end{center}
% Here, we see there exist 2 different cycles involving $M_2 \& M_3$ separately. Therefore, $M_2:W_1$, and $M_3:W_2$ gets paired.
% \\  \\
% The only women left unmatched is $W_3$, so \textbf{$M_1$ gets paired with $W_3$}.\\ \\
% \\
% $\therefore$ In all three cases, we see that if $M_1$ gets his third preference in DA, then \textbf{$M_1$ will always get the third preference using TTC}.

%\end{proof}
%\end{document}